\documentclass[
	% -- opções da classe memoir --
	12pt,				% tamanho da fonte
	openright,			% capítulos começam em pág ímpar (insere página vazia caso preciso)
	twoside,			% para impressão em recto e verso. Oposto a oneside
	a4paper,			% tamanho do papel.
	% -- opções da classe abntex2 --
	%chapter=TITLE,		% títulos de capítulos convertidos em letras maiúsculas
	%section=TITLE,		% títulos de seções convertidos em letras maiúsculas
	%subsection=TITLE,	% títulos de subseções convertidos em letras maiúsculas
	%subsubsection=TITLE,% títulos de subsubseções convertidos em letras maiúsculas
	% -- opções do pacote babel --
	english,			% idioma adicional para hifenização
	french,				% idioma adicional para hifenização
	spanish,			% idioma adicional para hifenização
	brazil				% o último idioma é o principal do documento
	]{abntex2}

% ---
% Pacotes básicos
% ---
\usepackage{lmodern}			% Usa a fonte Latin Modern
\usepackage[T1]{fontenc}		% Selecao de codigos de fonte.
\usepackage[utf8]{inputenc}		% Codificacao do documento (conversão automática dos acentos)
\usepackage{lastpage}			% Usado pela Ficha catalográfica
\usepackage{indentfirst}		% Indenta o primeiro parágrafo de cada seção.
\usepackage{color}				% Controle das cores
\usepackage{graphicx}			% Inclusão de gráficos
\usepackage{microtype} 			% para melhorias de justificação
\usepackage{amsmath}      % matemática
\usepackage{amssymb}      % também matemática
\usepackage{amsthm}       % idem
\usepackage{hyperref}     % para referenciar
\usepackage{tikz}
\usepackage{caption}
\usepackage[normalem]{ulem}
% ---

\makeatletter
\def\UL@putbox{\ifx\UL@start\@empty \else % not inner
  \vrule\@width\z@ \LA@penalty\@M
  {\UL@skip\wd\UL@box \UL@leaders \kern-\UL@skip}%
    \phantom{\box\UL@box}%
  \fi}
\makeatother

% ---
% Pacotes de citações
% ---
\usepackage[brazilian,hyperpageref]{backref}	 % Paginas com as citações na bibl
\usepackage[alf]{abntex2cite}	% Citações padrão ABNT

% ---
% CONFIGURAÇÕES DE PACOTES
% ---

% ---
% Configurações do pacote backref
% Usado sem a opção hyperpageref de backref
\renewcommand{\backrefpagesname}{Citado na(s) página(s):~}
% Texto padrão antes do número das páginas
\renewcommand{\backref}{}
% Define os textos da citação
\renewcommand*{\backrefalt}[4]{
	\ifcase #1 %
		Nenhuma citação no texto.%
	\or
		Citado na página #2.%
	\else
		Citado #1 vezes nas páginas #2.%
	\fi}%
% ---

% ---
% Criando alguns comandos de centralização no modo de matemática
% ---

\def\clap#1{\hbox to 0pt{\hss#1\hss}}
\def\mathllap{\mathpalette\mathllapinternal}
\def\mathrlap{\mathpalette\mathrlapinternal}
\def\mathclap{\mathpalette\mathclapinternal}
\def\mathllapinternal#1#2{%
\llap{$\mathsurround=0pt#1{#2}$}}
\def\mathrlapinternal#1#2{%
\rlap{$\mathsurround=0pt#1{#2}$}}
\def\mathclapinternal#1#2{%
\clap{$\mathsurround=0pt#1{#2}$}}

% ---
% Criando comando de definições e teoremas
% ---

\newtheorem{teorema}{Teorema}[chapter]
\newtheorem{definicao}{Definição}[chapter]
\newtheorem{corolario}{Corolário}[chapter]
\newtheorem{lema}{Lema}[chapter]

% ---
% Criando símbolos de matemática
% ---

\newcommand{\espaco}{\hspace{5em}}
\newcommand{\R}{\mathbb{R}}
\newcommand{\solucao}{\Psi(t, \mathring{p})}
\newcommand{\pbola}{\mathring{p}}
\newcommand{\pbarra}{\overline{p}}
\newcommand{\Xchapeu}{\hat{X}}
\newcommand{\Xchapeuik}{\hat{X}^i_k}
\newcommand{\Xbola}{\mathring{X}}
\newcommand{\xbola}{\mathring{x}}
\newcommand{\Xbolaik}{\mathring{X}^i_k}
\newcommand{\Xik}{X^i_k}
\newcommand{\xik}{x^i_k}
\newcommand{\xbarra}{\overline{x}}
\newcommand{\xbarraik}{\overline{x}^i_k}
\newcommand{\xij}{x^i_j}
\newcommand{\somamj}{\sum^m_{j=1}}
\newcommand{\somamk}{\sum^m_{k=1}}
\newcommand{\somani}{\sum^n_{i=1}}
\newcommand{\precos}{p_1, \ldots, p_m}
\newcommand{\U}{\mathcal{U}}
\newcommand{\Fi}{\varphi}
\newcommand{\Pbarra}{\overline{P}}
\newcommand{\Ubarrao}{\overline{U}_0}
\newcommand{\B}{\boldsymbol{B}}
\newcommand{\A}{\boldsymbol{A}}
\newcommand{\I}{\boldsymbol{I}}
\newcommand{\Mbarra}{\overline{M}}

% ---
% Informações de dados para CAPA e FOLHA DE ROSTO
% ---
\titulo{ESTABILIDADE EM EQUILÍBRIO WALRASIANO: WEINTRAUB E OS MÉTODOS DE LIAPUNOV
}
\autor{MARCELO AUGUSTO DONCATTO GELATI}
\local{Porto Alegre}
\data{2018}
\orientador{Jorge Paulo de Araújo}
\instituicao{%
  UNIVERSIDADE FEDERAL DO RIO GRANDE DO SUL
  \par
  FACULDADE DE CIÊNCIAS ECONÔMICAS}
\tipotrabalho{Trabalho de Conclusão de Curso (Graduação}
% O preambulo deve conter o tipo do trabalho, o objetivo,
% o nome da instituição e a área de concentração
\preambulo{Trabalho de conclusão submetido ao Curso de
Graduação em Ciências Econômicas da Faculdade de
Ciências Econômicas da UFRGS, como requisito
parcial para obtenção do título Bacharel em
Economia.}
% ---


% ---
% Configurações de aparência do PDF final

% alterando o aspecto da cor azul
\definecolor{blue}{RGB}{41,5,195}

% informações do PDF
\makeatletter
\hypersetup{
     	%pagebackref=true,
		pdftitle={\@title},
		pdfauthor={\@author},
    	pdfsubject={\imprimirpreambulo},
	    pdfcreator={LaTeX with abnTeX2},
		pdfkeywords={abnt}{latex}{abntex}{abntex2}{trabalho acadêmico},
		colorlinks=true,       		% false: boxed links; true: colored links
    	linkcolor=blue,          	% color of internal links
    	citecolor=blue,        		% color of links to bibliography
    	filecolor=magenta,      		% color of file links
		urlcolor=blue,
		bookmarksdepth=4
}
\makeatother
% ---

% ---
% Espaçamentos entre linhas e parágrafos
% ---

% O tamanho do parágrafo é dado por:
\setlength{\parindent}{1.3cm}

% Controle do espaçamento entre um parágrafo e outro:
\setlength{\parskip}{0.2cm}  % tente também \onelineskip

% ---
% compila o indice
% ---
\makeindex
% ---

% ----
% Início do documento
% ----
\begin{document}

% Seleciona o idioma do documento (conforme pacotes do babel)
%\selectlanguage{english}
\selectlanguage{brazil}

% Retira espaço extra obsoleto entre as frases.
\frenchspacing

% ----------------------------------------------------------
% ELEMENTOS PRÉ-TEXTUAIS
% ----------------------------------------------------------
% \pretextual

% ---
% Capa
% ---
\imprimircapa
% ---

% ---
% Folha de rosto
% (o * indica que haverá a ficha bibliográfica)
% ---
\imprimirfolhaderosto*
% ---

% ---
% Inserir a ficha bibliografica
% ---

% Isto é um exemplo de Ficha Catalográfica, ou ``Dados internacionais de
% catalogação-na-publicação''. Você pode utilizar este modelo como referência.
% Porém, provavelmente a biblioteca da sua universidade lhe fornecerá um PDF
% com a ficha catalográfica definitiva após a defesa do trabalho. Quando estiver
% com o documento, salve-o como PDF no diretório do seu projeto e substitua todo
% o conteúdo de implementação deste arquivo pelo comando abaixo:
%
% \begin{fichacatalografica}
%     \includepdf{fig_ficha_catalografica.pdf}
% \end{fichacatalografica}

%\begin{fichacatalografica}
%	\sffamily
%	\vspace*{\fill}					% Posição vertical
%	\begin{center}					% Minipage Centralizado
%	\fbox{\begin{minipage}[c][8cm]{13.5cm}		% Largura
%\small
%	\imprimirautor
%	%Sobrenome, Nome do autor
%
%	\hspace{0.5cm} \imprimirtitulo  / \imprimirautor. --
%	\imprimirlocal, \imprimirdata-
%
%	%\hspace{0.5cm} \pageref{LastPage} p. : il. (algumas color.) ; 30 cm.\\
%
%	\hspace{0.5cm} \imprimirorientadorRotulo~\imprimirorientador\\
%	\hspace{0.5cm}
%	\parbox[t]{\textwidth}{\imprimirtipotrabalho~--~\imprimirinstituicao,
%
%	\hspace{0.5cm}
%		1. Palavra-chave1.
%		2. Palavra-chave2.
%		2. Palavra-chave3.
%		I. Orientador.
%		III. Faculdade de xxx.
%		IV. Título
%	\end{minipage}}
%% ---

% ---
% Inserir folha de aprovação
% ---

% Isto é um exemplo de Folha de aprovação, elemento obrigatório da NBR
% 14724/2011 (seção 4.2.1.3). Você pode utilizar este modelo até a aprovação
% do trabalho. Após isso, substitua todo o conteúdo deste arquivo por uma
% imagem da página assinada pela banca com o comando abaixo:
%
% \includepdf{folhadeaprovacao_final.pdf}
%
\begin{folhadeaprovacao}

  \begin{center}
    {\ABNTEXchapterfont\large\imprimirautor}

    \vspace*{\fill}\vspace*{\fill}
    \begin{center}
      \ABNTEXchapterfont\bfseries\Large\imprimirtitulo
    \end{center}
    \vspace*{\fill}

    \hspace{.45\textwidth}
    \begin{minipage}{.5\textwidth}
        \imprimirpreambulo
    \end{minipage}%
    \vspace*{\fill}
   \end{center}

   Trabalho aprovado. \imprimirlocal, \uline{04} de \uline{dezembro} de 2018:

   \assinatura{\textbf{\imprimirorientador} \\ Orientador}
   \assinatura{\textbf{Marcelo de Carvalho Griebeler} \\ Convidado 1}
   \assinatura{\textbf{Sérgio Marley Modesto Monteiro} \\ Convidado 2}
   %\assinatura{\textbf{Professor} \\ Convidado 3}
   %\assinatura{\textbf{Professor} \\ Convidado 4}

   \begin{center}
    \vspace*{0.5cm}
    {\large\imprimirlocal}
    \par
    {\large\imprimirdata}
    \vspace*{1cm}
  \end{center}

\end{folhadeaprovacao}
% ---

% ---
% Dedicatória
% ---
\begin{dedicatoria}
   \vspace*{\fill}
   \centering
   \noindent
   \textit{ Este trabalho é dedicado a todos aqueles que\\
   me fizeram uma pessoa melhor.} \vspace*{\fill}
\end{dedicatoria}
% ---

% ---
% Agradecimentos
% ---
\begin{agradecimentos}

Faço meus agradecimentos em ordem cronológica.

Agradeço ao meu pai, por ter me estimulado desde muito cedo.
Agradeço à minha mãe, por ter me fornecido um vasto capital cultural.
Apesar dos desentendimentos, boa parte de minha trajetória foi definida por
vocês e estou muito satisfeito com ela.

Agradeço ao meu amigo Maurício, pelas discussões intelectuais realizadas em todas as vezes
que saíamos. Todas nossas conversas faziam com que eu tivesse vontade de continuar aprendendo
e ampliaram minhas áreas de interesse. Sei que sou uma pessoa muito menos limitada
hoje por causa de suas provocações.

Um de meus agradecimentos principais vai ao meu padrinho acadêmico, Fernando Sabino.
Enquanto meus pais definiram minha trajetória pré-acadêmica, você me orientou
para todas ótimas decisões que tomei ao entrar na faculdade. É de coração mesmo
que agradeço a todas conversas, observações, informações e a tudo mais que você
disponibilizou para me ajudar. Eu não teria chegado tão longe, não fosse por isso.

Agradeço ao chefe do meu primeiro estágio, Hudson. Todo tempo e informação que você
concedeu para que eu me qualificasse foram de extrema valia e me engrandeceram
significativamente.

Outro de meus agradecimentos principais vai a meu amigo Bruno. As linhas que
tenho que usar para lhe agradecer não conseguiriam conter tudo o que eu precisaria escrever.
Por isso agradeço a tudo. A tudo mesmo.

Apesar de a qualidade do curso em geral ter sido baixa, tive alguns
ótimos professores. Dentre estes, gostaria de agradecer principalmente ao professor Horn,
ao professor Monteiro, ao professor Jorge e ao professor Hélio. Suas aulas e personalidades
foram inspiradoras. Agradeço também ao professor Griebeler que, apesar de não ter sido
meu professor, teve paciência de ouvir minhas maluquices e me auxiliou quando
fui seu bolsista.

Por fim, meu maior agradecimento vai à minha companheira, Daiane.
Obrigado por ter me tornado mais humano. Obrigado, principalmente,
por ter me tornado pleno. Seu apoio e carinho diários me fizeram descobrir que,
independente do que eu faça, o que importa é que eu me sinta feliz. E eu me sinto
feliz com você. Obrigado.

\end{agradecimentos}
% ---

% ---
% Epígrafe
% ---
\begin{epigrafe}
    \vspace*{\fill}
	\begin{flushright}
		\textit{"I should like to say two things, one intellectual and one moral. \\
						The intellectual thing I should want to say is this: \\
						when you are studying any matter, or considering any philosophy, \\
						ask yourself only what are the facts and what is the truth that the facts bear out.\\
						Never let yourself be diverted either by what you wish to believe, \\
						or by what you think would have beneficent social effects if it were believed. \\
						But look only, and solely, at what are the facts. \\
						The moral thing I should wish to say is very simple \\
						I should say love is wise, hatred is foolish.\\
						In this world which is getting more and more closely interconnected \\
						we have to learn to tolerate each other, we have to learn \\
						to put up with the fact that some people say things that we don't like. \\
						We can only live together in that way and if we are to live together \\
						and not die together we must learn a kind of charity and a kind of \\
						tolerance which is absolutely vital to the continuation of human life on this planet".\\
						(Bertrand Russell)}
	\end{flushright}
\end{epigrafe}
% ---

% ---
% RESUMOS
% ---

% resumo em português
\setlength{\absparsep}{18pt} % ajusta o espaçamento dos parágrafos do resumo
\begin{resumo}

Esta monografia confronta a tese de \citeonline{weintraub1991} acerca do uso dos métodos
de Liapunov para o estudo da estabilidade do equilíbrio walrasiano.
Weintraub afirma que foi o artigo de \citeonline{bushawclower1954}
que introduziu os métodos para a comunidade acadêmica. Este trabalho
faz uma revisão da literatura sobre estabilidade em equilíbrio geral e mostra que
os métodos eram conhecidos antes da publicação de Clower e Bushaw.

 \textbf{Palavras-chave}: Equilíbrio geral. Estabilidade. Equações Diferenciais.

 Classificação JEL: B21; C26; D50.
\end{resumo}

% resumo em inglês
\begin{resumo}[Abstract]
 \begin{otherlanguage*}{english}

The present Bachelor thesis deals with Weintraub's (1991) commentary on the use of
Liapunov's methods for studying stability on a general equilibrium framework.
Weintraub points out that it was the work of Clower and Bushaw (1954) that
first introduced Liapunov's methods to the scientific community. This work
presents a literature review about stability in general equilibrium and then
shows that the methods were known before Clower and Bushaw's publication.

   \textbf{Keywords}: General Equilibrium. Stability. Differential Equations.

	 JEL classification: B21; C26; D50.
 \end{otherlanguage*}
\end{resumo}
% ---

% ---
% inserir lista de ilustrações
% ---
%\pdfbookmark[0]{\listfigurename}{lof}
%\listoffigures*
%\cleardoublepage
% ---

% ---
% inserir lista de tabelas
% ---
%\pdfbookmark[0]{\listtablename}{lot}
%\listoftables*
%\cleardoublepage
% ---

% ---
% inserir lista de abreviaturas e siglas
% ---
% \begin{siglas}
%  \item[EG] Equilíbrio Geral
%\end{siglas}
% ---

% ---
% inserir lista de símbolos
% ---
%\begin{simbolos}
%  \item[$\in$] Pertence
%\end{simbolos}
% ---

% ---
% inserir o sumario
% ---
\pdfbookmark[0]{\contentsname}{toc}
\tableofcontents*
\cleardoublepage
% ---



% ----------------------------------------------------------
% ELEMENTOS TEXTUAIS
% ----------------------------------------------------------
\textual

% ----------------------------------------------------------
% Introdução (exemplo de capítulo sem numeração, mas presente no Sumário)
% ----------------------------------------------------------
\chapter*[Introdução]{Introdução}
\addcontentsline{toc}{chapter}{Introdução}
% ----------------------------------------------------------

No livro de \citeonline{weintraub1991} há uma explicação de
como se deu a transformação da economia pré II Guerra Mundial para uma ciência
matematizada. Em meio a esta investigação, há uma discussão sobre a introdução
dos métodos de Liapunov em economia, especificamente nos trabalhos sobre
estabilidade em um sistema walrasiano. Nesta parte, Weintraub afirma que foi o
trabalho de \citeonline{bushawclower1954} que introduziu o conceito de
estabilidade em Liapunov na ciência econômica.

O critério usado para selecionar quem foram os pioneiros não é
\textit{stricto senso} quem foram os primeiros a utilizar Liapunov em um trabalho de
economia, pois Weintraub refere-se a um artigo de \citeyear{yasui1950}, escrito em
japonês por \citeauthoronline{yasui1950} que faz uso das funções de Liapunov. Assim,
Weintraub conclui que

\begin{citacao}[english]
	This said, it must be recognized that Yasui did not use Liapunov theory
	to "solve" a problem previously unsolved. He used the theory to simplify
	previous results. Further, his connection to the developing literature was
	remote in terms of distance and influence. His was not the work that
	introduced Liapunov theory to economists who were linked to the line of
	papers that resulted in the articles of Arrow, Block, and Hurwicz. \cite[p. 89]{weintraub1991}
\end{citacao}

Portanto, o que interessa a Weintraub é quem introduziu os métodos diretamente
para a comunidade científica, que neste caso seriam Clower e Bushaw. No entanto,
se desenvolveu uma consciência dentro da própria comunidade de que os
responsáveis por isso foram \citeonline{arrowhurwickz1958}. Segundo
Weintraub, o motivo principal é que na \textit{survey} clássica sobre
estabilidade walrasiana de \citeonline{negishi1962} \textbf{não} há quaisquer
referências ao trabalho de Clower e Bushaw, inclusive quando Negishi refere-se
explicitamente aos métodos de Liapunov.

Citando Weintraub,

\begin{citacao}[english]
[...] that judgment [o de Negishi] led to a belief that the Liapunov methods
were first introduced into the economics literature by Arrow and Hurwicz in
1958,  when in fact they had been introduced earlier. [...] More
important, however, is the fact that Clower and Bushaw explicitly used,
cited, and discussed the second or indirect Liapunov technique for demonstrating
asymptotic stability of the competitive equilibrium.
\cite[p. 136-7]{weintraub1991}
\end{citacao}

Independentemente de considerações sobre a intencionalidade de Negishi ao não
incluir o artigo daqueles economistas em sua \textit{survey}, meu trabalho
mostrará que a afirmação de Weintraub está equivocada, pois existe um trabalho
anterior ao de Clower e Bushaw, a saber \citeonline{arrowhurwicz1951}, que
utiliza o segundo método de Liapunov.

No entanto, Arrow e Hurwicz utilizam os métodos de Liapunov, mas não o
discutem explicitamente. Nessas linhas, ainda poderíamos considerar válido o argumento
de Weintraub. Apesar disso, ainda não cabe atribuir a Clower e Bushaw o
crédito. Feita a leitura dos artigos de
\citeonline{arrowhurwickz1958} e \citeonline{bushawclower1954}, constatei que
enquanto as funções de Liapunov são uma ferramenta central no trabalho de
Arrow e Hurwicz, Clower e Bushaw referem-se a ela apenas duas vezes: uma como
demonstração \textbf{alternativa} a um teorema já provado e outra em uma nota
de rodapé. O julgamento de Weintraub é feito no sentido de que os autores
foram ignorados mesmo após sua inédita contribuição. Ora, a contribuição não
pode ser inovadora se nem os autores a consideram dessa maneira. Trata-se
meramente de um exercício matemático. Diante dessas informações, a tese de Weintraub
continua sendo difícil de ser sustentada.

Minha investigação analisará esse problema de mais perto e descobrirá
quais foram as contribuições de fato em cada um dos artigos. A importância dela
é desmistificar uma ideia que está distante da realidade, já que
algumas teorias são criadas e perpetuadas sem a devida verificação. Descobri,
por exemplo, outros autores da história do pensamento econômico que citam a
tese de Weintraub. Eles dizem que:

\begin{citacao}[english]
[a]s pointed out by E. R. Weintraub (1991 :136–37),
the Bushaw-Clower article pioneered the Liapunov technique for demonstrating
that the addition of stock-adjustment mechanisms could led to asymptotic
stability of the competitive equilibrium, which was not recognized in the
literature at the time.
\cite[p. 47]{backhouseboianovsky2013}
\end{citacao}

Diante disso, é fundamental mostrar que há problemas na tese de Weintraub.

Para demonstrar meus argumentos, divido a monografia nas seguintes partes.
Na parte I discuto as origens do estudo moderno sobre estabilidade
walrasiana, começando com Hicks e passando pelas contribuições de Samuelson e
de um resultado clássico de Metzler.
Na parte II exponho os métodos de Liapunov e descrevo detalhadamente o paper de
Clower e Bushaw, mostrando exatamente onde os autores utilizaram tais
ferramentas.
Por fim, na parte III analiso em detalhes o artigo de Arrow e Hurwicz de 1958 e
faço um breve comentário sobre o artigo de 1951.

% PARTE SOBRE HICKS, SAMUELSON E METZLER
% ----------------------------------------------------------
\part{Hicks, Samuelson e Metzler}
% ----------------------------------------------------------
% ----------------------------------------------------------


% ---
% Capítulo de Hicks e Samuelson
% ---
\chapter{As estabilidades perfeita e dinâmica de Hicks e Samuelson} \label{hickssamuelson}
% ---

\section{Estabilidade no sentido de Hicks}

É no capítulo 5 de \citeonline{hicks1939} que o autor propõe algumas definições
de estabilidade em um sistema competitivo. Suponha que há $m$ mercados.
Estabilidade no mercado $k$, no sentido hicksiano, significa que quando o preço
do bem $k$ cai a demanda por $k$ será maior que a oferta (e vice-versa).
No entanto, duas situações podem ocorrer:

\begin{enumerate}
	\item a estabilidade se verifica após os outros $m - 1$ mercados ajustarem seus
	preços;
	\item a estabilidade se verifica após termos fixado o preço de qualquer subconjunto
	dos outros $m-1$ mercados \footnote{Note que por essa definição podemos fixar o preço de \textit{todos} os outros $m-1$ mercados.}.
\end{enumerate}

Caso o mercado seja estável no primeiro caso mas não no segundo, dizemos que ele
é \textit{imperfeitamente estável}. Se o mercado cumpre ambas condições
\footnote{É evidente que 2 $\implies$ 1. Portanto, é redundante especificar que o mercado
deve cumprir as duas condições.}, dizemos que é \textit{perfeitamente estável}.

Apesar de Hicks não ter formalizado matematicamente suas definições, faço isso
em seguida. A motivação vem de \citeonline{metzler1945} e nos ajudará a
criar a intuição para entender resultados posteriores.

Definindo a função excesso de demanda do bem $k$

\begin{equation}
	x_k(\precos) = D_k(\precos) - S_k(\precos)
\end{equation}

em que $D_k$ é a demanda pelo bem $k$ e $S_k$ é a oferta pelo bem $k$.

A estabilidade no mercado $k$ é definida pela condição de equilíbrio parcial

\begin{equation} \label{estab}
	\frac{dx_k}{dp_k} < 0.
\end{equation}

Para que o mercado seja imperfeitamente estável, \ref{estab} deve se verificar
após todos os outros mercados terem se ajustado, isto é,

\begin{equation}
	\frac{d}{dp_k}[x_j(p_1(p_k), \ldots, p_m(p_k))] = 0 \espaco para \; j \neq k.
\end{equation}

Montando o sistema para os $m$ mercados e fazendo a regra da cadeia, temos

\begin{equation} \label{sistimperf}
	\begin{aligned}
		& \frac{dx_1}{dp_k} = \frac{dx_1}{dp_1}\frac{dp_1}{dp_k} + \ldots + \frac{dx_1}{dp_k} + \ldots + \frac{dx_1}{dp_m}\frac{dp_m}{dp_k} = 0 \\
		& \vdots \\
		& \frac{dx_k}{dp_k} = \frac{dx_k}{dp_1}\frac{dp_1}{dp_k} + \ldots + \frac{dx_k}{dp_k} + \ldots + \frac{dx_k}{dp_m}\frac{dp_m}{dp_k}     \\
    & \vdots \\
	  & \frac{dx_m}{dp_k} = \frac{dx_m}{dp_1}\frac{dp_1}{dp_k} + \ldots + \frac{dx_m}{dp_k} + \ldots + \frac{dx_m}{dp_m}\frac{dp_m}{dp_k} = 0.
	\end{aligned}
\end{equation}

Matricialmente:

\begin{equation} \label{matrizimperf}
	\begin{pmatrix}
		\frac{dx_1}{dp_1} & \ldots & \frac{dx_1}{dp_m} \\
		\vdots \\
		\frac{dx_k}{dp_1} & \ldots & \frac{dx_k}{dp_m} \\
		\vdots \\
		\frac{dx_m}{dp_1} & \ldots & \frac{dx_m}{dp_m}
	\end{pmatrix} \:
	\begin{pmatrix}
		\frac{dp_1}{dp_k} \\
		\vdots \\
		1 \\
		\vdots \\
		\frac{dp_m}{dp_k}
	\end{pmatrix} \; = \;
	\begin{pmatrix}
		0 \\
		\vdots \\
		\frac{dx_k}{dp_k} \\
		\vdots \\
		0
	\end{pmatrix}
	.
\end{equation}

Por Cramer, obtemos

\begin{equation} \label{cramerimperf}
	1 =
	\frac{\begin{vmatrix}
		\frac{dx_1}{dp_1} & \ldots & 0 & \ldots & \frac{dx_1}{dp_m} \\
		\vdots \\
		\frac{dx_k}{dp_1} & \ldots & \frac{dx_k}{dp_k} & \ldots & \frac{dx_k}{dp_m} \\
		\vdots \\
		\frac{dx_m}{dp_1} & \ldots & 0 & \ldots & \frac{dx_m}{dp_m} \\
	\end{vmatrix}}
	{\begin{vmatrix}
		\frac{dx_1}{dp_1} & \ldots & \frac{dx_1}{dp_k} & \ldots & \frac{dx_1}{dp_m} \\
		\vdots \\
		\frac{dx_k}{dp_1} & \ldots & \frac{dx_k}{dp_k} & \ldots & \frac{dx_k}{dp_m} \\
		\vdots \\
		\frac{dx_m}{dp_1} & \ldots & \frac{dx_m}{dp_k} & \ldots & \frac{dx_m}{dp_m} \\
	\end{vmatrix}} \; = \;
   \frac{
	\begin{vmatrix}
		A(k)
	\end{vmatrix}}
	{\begin{vmatrix}
		A
	\end{vmatrix}}.
\end{equation}

Se utilizarmos a expansão de Laplace em $A(k)$ na coluna $k$, obteremos a matriz
$A(k)$ sem a linha e sem a coluna $k$ - que chamarei de $A(k|k)$ - multiplicada por $\frac{dx_k}{dp_k}$.
Assim, obtemos

\begin{equation} \label{laplaceimperf}
	1 =
	\frac{dx_k}{dp_k}
	\frac{\begin{vmatrix}
		A(k|k)
	\end{vmatrix}}
	{\begin{vmatrix}
		A
	\end{vmatrix}}
	\implies
	\frac{dx_k}{dp_k} =
	\frac{\begin{vmatrix}
		A
	\end{vmatrix}}
	{\begin{vmatrix}
		A(k|k)
	\end{vmatrix}}.
\end{equation}

Então, para que a condição \ref{estab} seja cumprida, o primeiro menor $k$
(isto é, $det[A(k|k)]$) tem que possuir sinal oposto a $det[A]$.

A formulação para a estabilidade perfeita é semelhante. Vamos supor que após uma alteração
no preço do bem $k$, sem perda de generalidade, deixamos o preço dos bens
$k + 1, \ldots, m$ fixos. Então, o sistema \ref{sistimperf} fica da seguinte maneira

\begin{equation} \label{sistperf}
	\begin{aligned}
		& \frac{dx_1}{dp_k} = \frac{dx_1}{dp_1}\frac{dp_1}{dp_k} + \ldots + \frac{dx_1}{dp_k}  = 0 \\
		& \vdots \\
		& \frac{dx_k}{dp_k} = \frac{dx_k}{dp_1}\frac{dp_1}{dp_k} + \ldots + \frac{dx_k}{dp_k}.
	\end{aligned}
\end{equation}

Matricialmente,

\begin{equation} \label{matrizperf}
	\begin{pmatrix}
		\frac{dx_1}{dp_1} & \ldots & \frac{dx_1}{dp_k} \\
		\vdots \\
		\frac{dx_k}{dp_1} & \ldots & \frac{dx_k}{dp_k} \\
	\end{pmatrix} \:
	\begin{pmatrix}
		\frac{dp_1}{dp_k} \\
		\vdots \\
		1 \\
	\end{pmatrix} \; = \;
	\begin{pmatrix}
		0 \\
		\vdots \\
		\frac{dx_k}{dp_k}
	\end{pmatrix}
	.
\end{equation}

Por Cramer,

\begin{equation} \label{cramerperf}
	1 =
	\frac{\begin{vmatrix}
		\frac{dx_1}{dp_1} & \ldots & 0  \\
		\vdots \\
		\frac{dx_k}{dp_1} & \ldots & \frac{dx_k}{dp_k}
	\end{vmatrix}}
	{\begin{vmatrix}
		\frac{dx_1}{dp_1} & \ldots & \frac{dx_1}{dp_k} \\
		\vdots \\
		\frac{dx_k}{dp_1} & \ldots & \frac{dx_k}{dp_k}
	\end{vmatrix}} \; = \;
   \frac{
	\begin{vmatrix}
		A^k(k)
	\end{vmatrix}}
	{\begin{vmatrix}
		A^k
	\end{vmatrix}}.
\end{equation}

Usando Laplace em $A^k(k)$ na coluna $k$, obteremos a mesma matriz sem a linha e
sem a coluna $k$, multiplicada por $\frac{dx_k}{dp_k}$. Isto é,

\begin{equation} \label{laplaceperf}
	1 =
	\frac{dx_k}{dp_k}
	\frac{\begin{vmatrix}
		A^k(k|k)
	\end{vmatrix}}
	{\begin{vmatrix}
		A^k
	\end{vmatrix}}
	\implies
	\frac{dx_k}{dp_k} =
	\frac{\begin{vmatrix}
		A^k
	\end{vmatrix}}
	{\begin{vmatrix}
		A^k(k|k)
	\end{vmatrix}}.
\end{equation}

Note que $A^k$ é o menor em que as linhas e colunas $k+1, \ldots, m$ foram retiradas.
$A^k(k|k)$ é o menor em que as linhas e colunas $k, \ldots, m$ foram retiradas.
Para que haja estabilidade, eles devem ter sinais opostos.
Portanto, estabilidade perfeita implica que os menores devem \textit{alternar}
os sinais \footnote{Isto é, os menores de ordem
ímpar devem ter sinal negativo e os de ordem par devem ter sinal positivo.}

\begin{equation} \label{hicksaltern}
	\frac{dx_k}{dp_k} < 0, \;
	\begin{vmatrix}
	 		\frac{dx_k}{dp_k} & \frac{dx_k}{dp_l} \\
			\frac{dx_l}{dp_k} & \frac{dx_l}{dp_l}
	 \end{vmatrix} > 0, \;
	 \begin{vmatrix}
			 \frac{dx_k}{dp_k} & \frac{dx_k}{dp_l} & \frac{dx_k}{dp_j} \\
			 \frac{dx_l}{dp_k} & \frac{dx_l}{dp_l} & \frac{dx_l}{dp_j} \\
			 \frac{dx_j}{dp_k} & \frac{dx_j}{dp_l} & \frac{dx_j}{dp_j}
	 \end{vmatrix} < 0, \; etc.,
\end{equation}

de tal forma que

\begin{equation} \label{hicksperf}
	sinal
\begin{vmatrix}
	A
\end{vmatrix} =
 sinal[(-1)^m].
\end{equation}

\section{As contribuições de Samuelson}

Infelizmente, a estabilidade no sentido de Hicks não é derivada de considerações
dinâmicas. O trabalho de Hicks é essencialmente estático. Para tentar entender
as relações entre estática comparativa e sistemas dinâmicos, \citeonline{samuelson1942}
cunha o termo \textit{princípio da correspondência}. Esse termo aos procedimentos
feitos por ele em um artigo anterior \cite{samuelson1941}.

Dentre as contribuições do artigo de \citeyear{samuelson1941}, destaco suas
definições de estabilidade. São elas:

\begin{enumerate}
	\item estabilidade do primeiro tipo "no grande" (ou global);
	\item estabilidade do primeiro tipo "no pequeno" (ou local);
	\item estabilidade do segundo tipo "no grande";
	\item estabilidade do segundo tipo "no pequeno".
\end{enumerate}

A estabilidade do primeiro significa que existe pelo menos um ponto de equilíbrio $\xbola$ tal que
algumas trajetórias convergem para esse ponto. A estabilidade \textbf{global}
especifica que independente do ponto inicial $\xbarra$, a trajetória atinge o ponto de equilíbrio $\xbola$.
Matematicamente, temos que

\begin{equation}
	\lim_{t \to \infty} x(t; \xbarra) = \xbola.
\end{equation}

Para que um ponto $\xbola$ seja \textbf{localmente} estável, deve existir uma vizinhança
$N(\xbola)$ de $\xbola$ tal que toda trajetória que inicie em algum ponto na vizinhança de $N(\xbola)$
converge para $\xbola$. Matematicamente, temos que

\begin{equation}
	\lim_{t \to \infty} x(t; \xbarra) = \xbola, \espaco \xbarra \in N(\xbola).
\end{equation}

Apesar de a estabilidade do segundo tipo não nos interessar, vale a pena
descrevê-la. Refere-se a um ponto de equilíbrio tal que a trajetória orbita em torno desse ponto,
nunca efetivamente convergindo para ele. As considerações sobre estabilidade local
e global se aplicam aqui da mesma maneira.

Neste trabalho uso o termo estabilidade dinâmica para me referir à estabilidade do
primeiro tipo.

Durante o início de 1940, algumas tentativas foram feitas para entender a relação
entre a estabilidade hicksiana e a estabilidade dinâmica. Por exemplo,
\citeonline{samuelson1941} mostra que a estabilidade imperfeita de Hicks não
era uma condição nem necessária nem suficiente para a estabilidade dinâmica.
Mostra também que a estabilidade perfeita não é uma condição necessária para
a estabilidade dinâmica. É somente em \citeyear{samuelson1944} que ele, com a
ajuda do matemático  Witold Hurewicz, mostra que a estabilidade perfeita
\textit{também} não é uma condição suficiente para a estabilidade dinâmica. O artigo dá um
exemplo em que há estabilidade hicksiana perfeita mas não há estabilidade dinâmica.
No próximo capítulo, mostro alguns exemplos em que essas situações acontecem.

Assim, os economistas descobriram que a estabilidade em Hicks não é uma condição
nem necessária e nem suficiente para a estabilidade dinâmica. No entanto, há
alguns casos em que é possível mostrar que as duas têm semelhanças. Portanto,
meu objetivo capítulo seguinte é explorar mais a fundo a relação entre estabilidade
dinâmica e hicksiana.

% ---
% Capítulo de Metzler
% ---
\chapter{Metzler e a integração das duas estabilidades} \label{metzler}
% ---

\section{Velocidades de ajustamento}

Conforme \citeonline{metzler1945}, é \citeonline{lange1944} que aponta pela primeira
vez que as formulações de Hicks, que vimos no capítulo \ref{hicks},
são instantâneas e não variam de mercado para mercado.
Ainda nesse livro, Lange monta um sistema com velocidades de
ajustamento.

Assim, temos o novo sistema dinâmico

\begin{equation}
	\frac{dp_i}{dt} = k_i x_i, \; k_i > 0 \espaco i = 1, \ldots, m
\end{equation}

em que a variação dos preços é proporcional
ao excesso de demanda pelo bem $x_i$, ponderado pelas velocidades de ajustamento $k_i$.

Há aqui uma leve mudança de notação. A partir de agora, utilizo o subíndice
$i$ para cada bem ao invés do antigo $k$. A letra $k$ agora representa a velocidade de ajustamento.

O exemplo a seguir vai trazer luz à relação entre as estabilidades hicksiana e dinâmica.

Imagine o seguinte sistema

\begin{equation}
	\begin{matrix}
		\frac{dp_1}{dt} = - k_1(p_1 - \pbola_1) - k_1(p_2 - \pbola_2) \\
		\frac{dp_2}{dt} = 2 k_2(p_1 - \pbola_1) + k_2(p_2 - \pbola_2).
	\end{matrix}
\end{equation}

Aqui, as funções $x_1(p_1, p_2)$ e $x_2(p_1, p_2)$ foram aproximadas pelos seu planos tangentes,
ou seja,

\begin{equation} \label{sist2}
	\begin{matrix}
		x_1(p_1, p_2) = x_1(\pbola_1, \pbola_2) + \frac{dx_1}{dp_1}(p_1 - \pbola_1) + \frac{dx_1}{dp_2}(p_2 - \pbola_2) \\
		x_2(p_1, p_2) = x_2(\pbola_1, \pbola_2) + \frac{dx_2}{dp_1}(p_1 - \pbola_1) + \frac{dx_2}{dp_2}(p_2 - \pbola_2).
	\end{matrix}
\end{equation}

Como $\pbola_1$ e $\pbola_2$ são pontos de equilíbrio, então $x_1(\pbola_1, \pbola_2)
= x_2(\pbola_1, \pbola_2) = 0$ e

\begin{equation}
	\begin{matrix}
		\frac{dx_1}{dp_1} = -1 & \frac{dx_1}{dp_2} = -1 \\
		\frac{dx_2}{dp_1} = 2 & \frac{dx_2}{dp_2} = 1.
	\end{matrix}
\end{equation}

Na forma matricial, temos

\begin{equation}
	k^TA =
	\begin{pmatrix}
		k_1 & k_2
	\end{pmatrix}
	\begin{pmatrix}
		-1 & -1 \\
		2 & 1
	\end{pmatrix} =
	\begin{pmatrix}
		-k_1 & -k_1 \\
		2k_2 & k_2
	\end{pmatrix}
\end{equation}

em que $k^T$ é o transposto do vetor coluna $k = (k_1, k_2)$.

Do ponto de vista hicksiano esse sistema não é imperfeitamente estável, pois
os dois menores principais de têm sinais diferentes (veja \ref{hicksaltern}).

O polinômio característico de $k^T A$ é

\begin{equation} \label{teste}
	p(\lambda) = \lambda^2 - (k_2 - k_1)\lambda + k_1k_2.
\end{equation}

Chamando as raízes de \ref{teste} de $\lambda_1$ e $\lambda_2$, temos que $\lambda_1 \lambda_2 = k_1 k_2 > 0$,
pois $k_1, k_2 > 0$. Portanto, as raízes devem ter o mesmo sinal. Além disso, temos que
$\lambda_1 + \lambda_2 = k_2 - k_1$. Portanto, se $k_1 > k_2 \implies \lambda_1 + \lambda_2 < 0
\implies$ os dois autovalores são negativos e o sistema é dinamicamente estável.
Se $k_1 < k_2$ o sistema não é estável.

Este exemplo mostra que mesmo que as condições de Hicks não estejam satisfeitas é
possível que o sistema seja dinamicamente estável e que a estabilidade depende também
das velocidades de ajustamento e não só dos $\frac{dx_i}{dp_j}$.

Assim, podemos interpretar a estabilidade de Hicks de outra maneira. Podemos imaginar
que ela é uma formulação que \textit{independe} das velocidades de ajustes.
Neste contexto, Metzler mostra que se um sistema é dinamicamente estável
para quaisquer $k_1 > 0, \ldots, k_m > 0$, então o sistema também é (perfeitamente) estável no sentido
de Hicks. Este é o resultado que mostrarei a seguir.

\section{Primeiro teorema de Metzler}

Vamos montar o sistema \ref{sist2} para
os $m$ mercados com as velocidades de ajustamento.

\begin{equation} \label{sistm}
		\frac{dp_i}{dt} = k_ia_{i1}(p_1 - \pbola_1) + \ldots + k_ia_{im}(p_m - \pbola_m) \espaco i = 1, \ldots, m
\end{equation}

em que $a_{ij} = \frac{dx_i}{dp_j}$.

O polinômio característico de \ref{sistm} é

\begin{equation} \label{poli}
	p(\lambda) =
	\begin{vmatrix}
		\lambda - k_1a_{11} & -k_1a_{12} & \ldots & -k_1a_{1m} \\
		-k_2a_{21} & \lambda - k_2a_{22} & \ldots & -k_2a_{2m} \\
		\vdots & \vdots & & \vdots \\
		-k_ma_{m1} & -k_m a_{m2} & \ldots & \lambda - k_ma_{mm}
	\end{vmatrix} =
	\begin{vmatrix}
		\lambda I - k^T A
	\end{vmatrix}
\end{equation}

Expandindo \ref{poli}, temos

\begin{equation} \label{policarac}
	p(\lambda) = \lambda^m + (-1)^1 S_1 \lambda^{m-1} + (-1)^2 S_2 \lambda^{m-2} + \ldots +
	(-1)^{m-1} S_{m-1} \lambda + (-1)^m S_m
\end{equation}

em que $S_i$, $i = 1, \ldots, m$, é o somatório de todos os menores principais
de ordem $i$ da matriz $k^T A$ .

Para que ele seja estável, as raízes $\lambda_m$ devem ter parte real negativa.
Agora note duas condições abaixo:

\begin{subequations}
\begin{equation} \label{condic1}
	p(0) = (-1)^m S_m = (-1)^m
	\begin{vmatrix}
		k^T A
	\end{vmatrix}.
\end{equation}

\begin{equation} \label{condic2}
	\lim_{\lambda \to \infty} p(\lambda) = \infty.
\end{equation}
\end{subequations}

Se  $(-1)^m$ e $|k^T A|$ tiverem sinais opostos, $p(0) < 0$ por \ref{condic1}.
Mas como \ref{condic2} nos garante que $p(\lambda) \to \infty$, o eixo $\lambda$
será cortado. Portanto, o comportamento de \ref{policarac} será (grosseiramente)
assim:

\newpage

\begin{center}
\begin{tikzpicture}[scale=3]
\draw[->] (0, 0) -- (1,0) node[right] {$\lambda$};
\draw[->] (0,-1) -- (0,1) node[above] {$p(\lambda)$};
\draw[blue] (0, -0.2) -- (1, 0.8);
\end{tikzpicture}
\captionof{figure}{Instabilidade do sistema \ref{sistm}}
\end{center}

Nesse caso, o sistema é instável, pois $p(\lambda) = 0$ para algum $\lambda$ com parte
real positiva.

Se $(-1)^m$ e $|k^T A|$ têm o mesmo sinal, o comportamento será desse tipo:

\begin{center}
\begin{tikzpicture}[scale=3]
\draw[->] (0, 0) -- (1,0) node[right] {$\lambda$};
\draw[->] (0,-1) -- (0,1) node[above] {$p(\lambda)$};
\draw[blue] (0, 0.1) -- (1, 1);
\end{tikzpicture}
\captionof{figure}{Estabilidade do sistema \ref{sistm}}
\end{center}

E não há raízes com parte real positiva. Portanto, o sistema é estável.
Mas essa é exatamente a última condição de estabilidade perfeita de Hicks (ver \ref{hicksperf})!
Agora, basta verificar as condições \ref{hicksaltern} e garantiremos que um sistema dinamicamente estável
atende as condições de Hicks.

Imagine que alguns mercados são muito inflexíveis em relação a outros, isto é,
suas velocidades de ajustamento são muito pequenas. Supondo que esses $m-n$ mercados não se ajustem, colocamos
suas velocidades de ajustamento $k_{n+1} = \ldots = k_m = 0$. Os outros $n$ mercados formam o sistema

\begin{equation} \label{sistn}
		\frac{dp_i}{dt} = k_ia_{i1}(p_1 - \pbola_1) + \ldots + k_ia_{in}(p_n - \pbola_n) \espaco i = 1, \ldots, n.
\end{equation}

Ora, repetindo os procedimentos realizados acima, descobirmos que o determinante desse
sistema deve ter o mesmo sinal de $(-1)^n$. Portanto, as condições de Hicks são satisfeitas
e ele é perfeitamente estável e temos o seguinte teorema:

\begin{teorema} \label{metzler1}
	Se um sistema é dinamicamente estável para quaisquer velocidades de ajustamento,
	então ele também é perfeitamente estável.
\end{teorema}

\section{Segundo teorema de Metzler}

Apesar de ser uma condição necessária, ela não é uma condição suficiente. O exemplo
a seguir ilustra isso.

Considere o sistema

\begin{equation}
	\begin{matrix}
		\frac{dp_1}{dt} =& -k_1(p_1 - \pbola_1) &- 0.8k_1(p_2 - \pbola_2) & \\
		\frac{dp_2}{dt} =&                      &- k_2(p_2 - \pbola_2)    &- 0.8k_2(p_3 - \pbola_3) \\
		\frac{dp_3}{dt} =& -10k_3(p1 - \pbola_1)&- k_3(p_2 - \pbola_2)   &- k_3(p_3 - \pbola_3)
	\end{matrix}
\end{equation}

e sua matriz

\begin{equation}
	\begin{pmatrix}
		-k_1   & -0.8k_1 & 0 \\
		0      & -k_2    & -0.8k_2 \\
		-10k_3 & -k_3    & -k_3
	\end{pmatrix}
	.
\end{equation}

Os menores principais de primeira ordem são todos negativos ($-k_1, -k_2, -k_3$), os de segunda
ordem são positivos ($0.2k_2k_3, k_1k_3, k_1k_2$) e o determinante da matriz é
negativo ($-6.6k_1k_2k_3$). Portanto, a matriz é perfeitamente estável.

Colocando $k_1 = 2, k_2 = 2, k_3 = 1$, temos o polinômio característico

\begin{equation}
	\lambda^3 + 5\lambda^2 + 6.4\lambda + 26.4
\end{equation}

com raízes $\lambda_{1, 2} \approx -0.095 \pm 2.34i$ e $\lambda_3 \approx -4.81$.
Como as partes reais são todas negativas, o sistema é dinamicamente estável.

Se fizermos $k_1 = 1, k_2 = 2, k_3 = 2$, entretanto, o sistema não será assintoticamente estável.
O polinômio característico é

\begin{equation}
	\lambda^3 + 5\lambda^2 + 4.8\lambda + 26.4
\end{equation}

e as raízes são $\lambda_{1, 2} \approx 0.039 \pm 2.28i$ e $\lambda_3 \approx - 5.08$.
Como as raízes complexas têm parte real positiva, o sistema não é estável.

Visto ess resultado, precisamos de uma condição mais forte para integrar as duas noções de
estabilidade. É aqui que Metzler dá início ao seu resultado principal.

Metzler supõe a condição de substituibilidade bruta entre todos os bens, isto é,

\begin{equation}
	\frac{dx_i}{dp_j} > 0 \espaco i \neq j
\end{equation}

Lançando mão dessa condição, Metzler mostra que a estabilidade perfeita no sentido
de Hicks é equivalente à estabilidade dinâmica.

Para demonstrar essa afirmação, Metzler introduz um sistema de equações em diferenças
que serve como auxiliar na demonstração

\begin{equation} \label{sistdif}
	\begin{matrix}
		y_1(t) =& (k_1a_{11} + 1)y_1(t - 1) &+& k_1a_{12}y_2(t-1) &+& \ldots &+& k_1a_{1m}y_m(t-1) \\
		y_2(t) =& k_2a_{21}y_1(t - 1) &+& (k_2a_{22} + 1)y_2(t-1) &+& \ldots &+& k_2a_{2m}y_m(t-1) \\
		\vdots \\
		y_m(t) =& k_ma_{m1}y_1(t - 1) &+& k_ma_{m2}y_2(t-1) &+& \ldots &+& (k_ma_{mm} + 1)y_m(t-1)
	\end{matrix}
\end{equation}

e sua matriz

\begin{equation} \label{matrizdif}
	B =
	\begin{pmatrix}
		k_1a_{11} + 1 & k_1a_{12} & \ldots & k_1a_{1m} \\
		k_2a_{21} & k_2a_{22} + 1 & \ldots & k_2a_{2m} \\
		\vdots \\
		k_ma_{m1} & k_ma_{m2} & \ldots & k_ma_{mm} + 1
	\end{pmatrix}.
\end{equation}

A primeira etapa da argumentação é mostrar que o sistema de equações em diferenças \ref{sistdif} é estável
se, e só se, o sistema \ref{sistm} é estável. Minha demonstração se desviará
levemente da demonstração de Metzler, por acreditar que é defeituosa
em um determinado procedimento \footnote{Agradeço ao orientador Jorge P. de Araújo por apontar
isso e pela sugestão de demonstração alternativa.}.

Montando o polinômio característico de \ref{sistdif}

\begin{equation} \label{polidif}
	q(\rho) =
	\begin{vmatrix}
		\rho - (k_1a_{11} + 1) & - k_1a_{12} & \ldots & - k_1a_{1m} \\
		- k_2a_{21} & \rho - (k_2a_{22} + 1) & \ldots & - k_2a_{2m} \\
		\vdots \\
		- k_ma_{m1} & - k_ma_{m2} & \ldots & \rho - (k_ma_{mm} + 1).
	\end{vmatrix}
\end{equation}

Se colocarmos $\rho - 1 = \lambda$, o polinômio característico \ref{polidif} é igual
ao polinômio \ref{poli}. Ou seja, $q(\rho) = q(\lambda + 1) = p(\lambda)$. Além disso,
se $\lambda_i$ é raíz de $p$, então $\lambda_i + 1$ é raíz de $q$ (e vice-versa).

\begin{lema}
	Se \ref{sistdif} é estável, então \ref{sistm} é estável.
\end{lema}

\begin{proof}
	Seja $\lambda_i + 1$ as raízes de $q(\lambda + 1)$. Para que haja estabilidade em
	\ref{sistdif} é preciso que as partes reais das raízes sejam em módulo menor que 1,
	isto é,

	\begin{equation}
		\begin{aligned}
		|Re(\lambda_i + 1)| < 1 \implies -1 < Re(\lambda_i + 1) < 1 \implies \\
		\implies -1 < Re(\lambda_i) + 1 < 1 \implies -2 < Re(\lambda_i) < 0.
	\end{aligned}
	\end{equation}

		Descobrimos que $\lambda_i$ tem parte real negativa e como $\lambda_i$ é raíz de $p$,
		\ref{sistm} é estável.
\end{proof}

\begin{lema}
	Se \ref{sistm} é estável, então \ref{sistdif} é estável.
\end{lema}

\begin{proof}
Como a estabilidade do sistema é independente das unidades de tempo, escolhemos
unidades convenientes de maneira que $0 < k_ia_{ii} + 1 < 1$. Além disso, como
os bens são substitutos brutos, $0 < k_ia_{ij} < 1$.

Dessa forma, a matriz \ref{matrizdif} tem todos os componentes positivos. Pelo teorema de Perron-Frobenius,
sabemos que existe um autovalor de B, $\lambda(B)$, tal que $\lambda(B) > 0$.
Se $\omega$ é outro autovalor de B, então $|\omega| < \lambda(B)$.

Vamos provar por contrapositivo. Suponhamos que \ref{sistdif} não é estável. Então, existe uma raiz $\mu$ de
$q(\lambda)$ tal que $|\mu| \geq 1$. Nesse caso, $q(\lambda)$ tem uma raiz real $\lambda(B) > |\mu| > 1$.

Assim, $\lambda(B) - 1 > 0$ é raíz de $p(\lambda)$ e, como ela é positiva,
o sistema \ref{sistm} não é estável.
\end{proof}

Com os procedimentos acima, chegamos no seguinte teorema.

\begin{teorema}
	O sistema \ref{sistm} é estável se, e só se, \ref{sistdif} é estável.
\end{teorema}

Agora, basta provar que a estabilidade de \ref{sistdif} é equivalente à estabilidade
perfeita de Hicks.

\begin{lema}
	Se \ref{sistdif} é perfeitamente estável no sentido de Hicks, então
	\ref{sistdif} é dinamicamente estável
\end{lema}

\begin{proof}
Expandindo o polinômio característico da matriz \ref{matrizdif}, temos

\begin{equation} \label{polidifexpand}
	\begin{aligned}
	q(\lambda) = p(\lambda - 1) = (\lambda - 1)^m + (-1)^1 S_1(\lambda - 1)^{m-1} + \ldots + \\
	+ (-1)^{m-1}S_{m-1}(\lambda - 1) + (-1)^m S_m.
	\end{aligned}
\end{equation}

Se \ref{sistdif} tem estabilidade perfeita no sentido de Hicks, os menores de
ordem $i$ têm o mesmo sinal de $(-1)^i$. Assim, \ref{polidifexpand} tem todos os coeficientes
positivos. Pela regra dos sinais de Descartes, não há nenhuma raiz real não negativa,
isto é, $\lambda - 1 < 0 \implies \lambda < 1$. Novamente, pelo teorema
de Perron-Frobenius, $|\omega| < \lambda(B) < 1$ para toda raiz $\omega$ de
\ref{polidif}. Como todas raízes estão dentro do círculo unitário, o sistema de equações em diferenças
é estável.
\end{proof}

Agora, basta provar a volta.

\begin{lema}
	 Se \ref{sistdif} é dinamicamente estável, então
		\ref{sistdif} é perfeitamente estável no sentido de Hicks.
\end{lema}

\begin{proof}
	Consideremos uma versão reduzida a $n$ mercados de \ref{sistdif}.

	\begin{equation} \label{sistdifreduz}
		\begin{matrix}
				\hat{y_1}(t) =& (k_1a_{11} + 1)\hat{y_1}(t - 1) &+& k_1a_{12}\hat{y_2}(t-1) &+& \ldots &+& k_1a_{1n}\hat{y_n}(t-1) \\
				\hat{y_2}(t) =& k_2a_{21}\hat{y_1}(t - 1) &+& (k_2a_{22} + 1)\hat{y_2}(t-1) &+& \ldots &+& k_2a_{2n}\hat{y_n}(t-1) \\
				\vdots \\
				\hat{y_n}(t) =& k_na_{n1}\hat{y_1}(t - 1) &+& k_na_{n2}\hat{y_2}(t-1) &+& \ldots &+& (k_na_{nn} + 1)\hat{y_n}(t-1)
		\end{matrix}
	\end{equation}

	Vamos mostrar que o sistema completo é estável somente se o sistema reduzido é estável.

	Se $y_i \geq \hat{y_i} > 0$ para os valores iniciais, então
	$y_i(t) \geq \hat{y_i}(t)$ para qualquer $t$. Provamos esta última afirmação por
	indução. Se $y_i(t-1) \geq \hat{y_i}(t-1)$ para todo $j = 1, \ldots, n$, então
	$y_i(t)$ é uma soma de termos $\geq 0$, dos quais $m$ termos estão presentes em
	$\hat{y_i}(t)$. Portanto, $y_i(t) \geq \hat{y_i}(t)$. Assim, se o sistema reduzido
	é instável, o sistema maior também é.

	Para que o sistema reduzido seja estável, é preciso que $(-1)^n$ tenha o mesmo sinal do
	determinante de ordem $n$

\begin{equation}
	\begin{vmatrix}
		a_{11} & \ldots & a_{1n} \\
		\vdots \\
		a_{n1} & \ldots & a_{nn}
	\end{vmatrix}
\end{equation}

	como vimos antes. Mas, novamente, essa é a condição de estabilidade perfeita em Hicks.

\end{proof}

Os dois lemas anteriores nos dão o seguinte teorema.

\begin{teorema}
	Se há substituibilidade bruta dos bens, a estabilidade perfeita no sentido de Hicks
	é equivalente a estabilidade dinâmica.
\end{teorema}

Terminada a parte I, iremos olhar para os métodos de Liapunov mais de perto.
O capítulo seguinte mostrará sobre o que se trata o segundo método de Liapunov
e em seguida conseguiremos analisar o paper de \citeonline{bushawclower1954}.

% ----------------------------------------------------------
% PARTE SOBRE CLOWER E BUSHAW
% ----------------------------------------------------------
\part{Liapunov, Clower e Bushaw}
% ------------------------------

% ---
% Capitulo de Liapunov
% ---
\chapter{Os métodos de Liapunov} \label{liapunov}
% ---

% ---
\section{Teorema de Liapunov}
% ---

Seja $\U \subseteq \R^n$ aberto, $X: \U \to \R^n$ um campo de vetores de classe
$C^\infty$, $x_0 \in \U$
e $X(x_0) = 0$. Com isso, consideremos o seguinte sistema dinâmico

\begin{equation}
	x' = X(x),
\end{equation}

isto é,

\begin{equation}
	X(x) =
	\begin{pmatrix}
		f_1(x) \\
		\vdots \\
		f_n(x)
	\end{pmatrix}
	\implies
	\begin{matrix}
		x'_1(t) = f_1(x_1(t), \ldots, x_n(t)) \\
		\vdots \\
		x'_n(t) = f_n(x_1(t), \ldots, x_n(t)). \\
	\end{matrix}
\end{equation}

Observe que $x'(t) = x_0$ é um equilíbrio estacionário para o sistema acima.

\begin{definicao}
	Um fluxo em $\U$ é uma aplicação $\Fi: \R \times \U \to \U$ tal que

	\begin{enumerate}
		\item $\Fi(0, x) = x$;
		\item $\Fi(t, \Fi(s, x)) = \Fi(t + s, x)$.
	\end{enumerate}
\end{definicao}

A ideia é que a trajetória do sistema $x' = X(x)$ define um fluxo $\Fi(t, x)$.
Em outras palavras, $\Fi(t, \xbarra) = x(t)$ em que $x(0) = \xbarra$.

\begin{definicao} \label{funcliapunov}
	Uma função de Liapunov para o campo $X$ em $x_0$ é uma função
	$V: U_0 \to \R$ de classe $C^1$, definida em uma vizinhança $U_0 \subseteq \U$ de $x_0$, tal que
 		\begin{enumerate}
			\item $V(x_0) = 0$, $V(x) > 0$, para $x \neq x_0$;
			\item $\nabla V(x) \cdot X \leq 0$ em $U_0$
		\end{enumerate}
\end{definicao}

Se a desigualdade é estrita no item 2, dizemos que é uma função de Liapunov estrita.

\begin{teorema}[\textit{Teorema de Liapunov}] \label{teorliapunov}
	Se existe uma função de Liapunov para o campo $X$ em $x_0$, então $x_0$ é um
	equilíbrio estável. Ademais, se existe uma função de Liapunov estrita, o equilíbrio
	será assintoticamente estável.
\end{teorema}

\begin{proof}
	Consideremos uma vizinhança aberta tal que o fecho compacto $\overline{U}_0$ de
	$U_0$ é tal que
		\begin{equation*}
			m = \min_{\partial \overline{U}_0} V(x)
		\end{equation*}

	Antes de continuar a demonstração, apenas clarifico que $\partial \overline{U}_0$
	refere-se à fronteira de $\overline{U}_0$. Isto é,

	\begin{equation}
		\partial \overline{U}_0 = \{x \in \R^n \; | \; \exists \{x_n\} \subseteq \overline{U}_0
			\; \text{com} \; x_n \to x \; \text{e} \; \exists \{y_n\} \subseteq \overline{U}_0^c \; \text{com} \;  y_n \to x\}.
	\end{equation}

	Na reta, por exemplo, a fronteira do conjunto $A = (0, 1)$ é $\partial A = \{0, 1\}$.

	Prosseguimos com a demonstração. Como $V$ é contínua e $V(x_0) = 0$, então podemos escolher uma vizinhança aberta
	$U_1 \subseteq U_0$, $x \in U_1$ e $V(x) < m$.

	Se $\Fi(t, x)$ é o fluxo determinado por $x' = X(x)$, então $V(\Fi(t, x)) < m,
	\; \forall t \geq 0$, pois $V$ é decrescente sobre as trajetórias de $x' = X$.

	Devemos mostrar que $\Fi(t, x) \in \overline{U}_0 \ \forall t \geq 0$.
	Vamos provar por absurdo. Se existe $t' > 0$
	tal que $\Fi(t', x) \notin \Ubarrao$, então há um $t''$ tal que $t' > t'' > 0$
	e $\Fi(t'', x) \in \partial \Ubarrao$ e portanto $V(\Fi(t'', x)) \geq m$,
	pois $m$ é o mínimo de V em $\partial \Ubarrao$. Mas isto é um absurdo, pois
	$V(\Fi(t, x)) < m \; \forall t \geq 0$.

	Logo, $\Fi(t, x_0) = x_0$ é uma solução estável. Nos resta provar agora o segundo caso.

	Suponhamos que exista uma trajetória $\Fi(t, x)$ tal que

	\begin{equation*}
		\lim_{t \to \infty} \Fi(t, x) \neq x_0.
	\end{equation*}

	Então, existe uma sequência $t_1 < t_2 < \ldots < t_n < \ldots$ tal que

	\begin{equation*}
		\lim_{n \to \infty} \Fi(t_n, x) \neq x_0 \espaco \text{e} \espaco \lim_{k \to \infty} \Fi(t_{n_k}, x) = y,
	\end{equation*}

	sendo a igualdade garantida pela compacidade de $\Ubarrao$ e pelo teorema de
	Bolzano-Weirstrass.

	Ademais, temos que

	\begin{equation*}
		\lim_{k \to \infty} \Fi(t_{n_k} + 1, x) = \lim_{k \to \infty} \Fi(1, \Fi(t_{n_k}, x)) = \Fi(1, y).
	\end{equation*}

	Pela continuidade de V,

	\begin{equation*}
		\begin{matrix}
			V(\Fi(t_{n_k}, x)) \to V(y) = V(\Fi(0, y)) \\
			V(\Fi(t_{n_k} + 1, x)) \to V(\Fi(1, y)),
		\end{matrix}
	\end{equation*}

	mas isso implica que

	\begin{equation*}
		V(\Fi(0, y)) = V(\Fi(1, y)),
	\end{equation*}

	o que é uma contradição, pois por hipótese V é estritamente decrescente sobre
	as trajetórias. Então

	\begin{equation*}
		\lim_{t \to \infty} \Fi(t, x) = x_0 \espaco \text{para} \ x \in U_1
	\end{equation*}

	e $x_0$ é um equilíbrio assintoticamente estável.

\end{proof}

	Procedemos agora para analisar o artigo de \citeonline{bushawclower1954}.

% ---
% Capitulo de Bushaw e Clower
% ---
\chapter{O sistema estoque-fluxo de Bushaw e Clower}
% ---

\section{Contribuições...?}

O artigo de \citeonline{bushawclower1954} é um pequeno desvio do caminho que
tracei até agora. Enquanto tudo que expus anteriormente remetia à pesquisa sobre estabilidade
de um equilíbrio geral, aqui os autores pretendem criar uma teoria geral de determinação
de preços em um modelo de estoque-fluxo. Citando a introdução de seu artigo,

\begin{citacao}[english]
	In recent years economists have devoted considerable attention to special
problems involving stock-flow relationship. Thus far, however, attempts to
study such relationships within the framework of the general theory of price
determination have been few in number and highly restricted in scope. The
object of the present paper is to fill at least part of this gap in the existing literature.
It is a study in the pure theory of market price determination in a stock-flow economy - an
economy in which the typical commodity is simultaneously produced, consumed, and
held for future disposal by economic units. \cite[p. 328]{bushawclower1954}
\end{citacao}

Logo na introdução já temos uma pista do motivo de \citeonline{negishi1962}
não ter citado Clower e Bushaw em sua \textit{survey}: o foco do artigo \textit{não}
era estudar a estabilidade de um equilíbrio geral. Além do mais, o modelo proposto pelos autores
difere tanto daquele proposto por Hicks e Samuelson quanto daquele proposto por
Arrow e Hurwicz (como veremos a seguir). Weintraub não percebe que, simplesmente,
esse artigo não foi relevante para a pesquisa da estabilidade do equilíbrio walrasiano.
Os autores propõem uma modelagem diferente do que havia à época
e a comunidade científica não incorporou suas contribuições.

Mas ainda assim, Weintraub se refere especificamente aos métodos de Liapunov
introduzidos pelos autores, não à contribuição do artigo em geral.
A investigação detalhada do artigo a seguir mostrará
que a contribuição deles (para a inserção dos métodos de Liapunov) também não foi tão
significante assim.

\section{Estoque e fluxo: definições}

Definimos a notação de antemão:

\begin{definicao} Para o restante do capítulo, consideramos que:
\begin{itemize}
	\item há $m$ mercados;
	\item $p_i$ é o preço do i-ésimo bem, $\precos$;
	\item $d_i(\precos)$ é a \textit{flow demand} (demanda corrente) e $s_i(\precos)$
	é a \textit{flow supply} (oferta corrente) do bem $i$;
	\item $D_i(\precos)$ é a demanda pelo estoque do bem $i$;
	\item $S_i \equiv S_i^0 + \displaystyle\int^t_{t_0}(s_i - d_i)dt$, $S_i$ é a oferta pelo
	estoque do bem $i$, em que $S_i^0$ é o estoque do bem $i$ no tempo $t_0$.
	\item $X_i(\precos; t) \equiv D_i(\precos) - S_i^0 + \displaystyle \int^t_{t_0}x_i(\precos)dt$, em que
	$X_i$ é o excesso de demanda por estoques e $x_i \equiv d_i - s_i$ é o excesso de demanda corrente.
\end{itemize}
\end{definicao}

Quando me refiro aos termos oferta/demanda corrente, quero dizer apenas que é a
oferta/demanda que ocorre em cada período $t$, sem depender das produções passadas.

Supomos mais uma vez que a variação dos preços é uma função
do excesso de demanda, mas desta vez incluímos além da produção corrente a produção de estoque, isto é,

\begin{equation} \label{funcp}
	\frac{dp_i}{dt} = f_i(x_i, X_i) \espaco i = 1, \ldots, m
\end{equation}

Faremos algumas suposições sobre o comportamento de \ref{funcp}:

\begin{subequations}

\begin{equation} \label{comport1}
		\bigg(\frac{\partial f_i}{\partial x_i}\bigg)^2 + \bigg(\frac{\partial f_i}{\partial X_i}\bigg)^2 > 0 \quad \text{e}
\end{equation}

\begin{equation} \label{comport2}
			f_i(0, 0) = 0.
\end{equation}
\end{subequations}

Em uma vizinhança dos pontos $\xbarra_i$ e $\overline{X}_i$ e expandindo em Taylor, temos que

\begin{equation}
	f_i \approx f_i(\xbarra_i, \overline{X}_i) + \frac{\partial f_i}{\partial x_i}(x_i - \xbarra_i)
	 	  + \frac{\partial f_i}{\partial X_i}(X_i - \overline{X}_i)
			= \alpha_i(x_i - \xbarra_i) + \beta_i(X_i - \overline{X}_i)
\end{equation}

e por \ref{comport1}, não pode ocorrer que $\alpha_i = 0$ e $\beta_i = 0$ simultaneamente.
Se colocarmos $\xbarra_i = \overline{X}_i = 0$, temos por \ref{comport2}

\begin{equation} \label{qi}
	f_i = \alpha_ix_i + \beta_iX_i \espaco i = 1, \ldots, m
\end{equation}

\begin{definicao}
	Se $\alpha_i = 0$, chamaremos o bem $i$ de estoque-orientado. Se $\beta_i = 0$,
	o bem $i$ será chamado de fluxo-orientado.
\end{definicao}

\section{O equilíbrio}

Suponhamos que na vizinhança de $p^0 = (p^0_1, \ldots, p^0_m)$ as funções
$x_i$ e $D_i$ admitam a seguinte expansão em séries de potências:

\begin{subequations}

	\begin{equation}
		x_i = x^0_i + \somamj a_{ij}(p_j - p^0_j) + [(p_1 - p^0_1), \ldots, (p_m - p^0_m)]
	\end{equation}

	\begin{equation}
		D_i = D^0_i + \somamj b_{ij}(p_j - p^0_j) + [(p_1 - p^0_1), \ldots, (p_m - p^0_m)]
	\end{equation}

\end{subequations}

O símbolo $[(p_1 - p^0_1), \ldots, (p_m - p^0_m)]$ indica alguma série de potências
nas variáveis $(p_1 - p^0_1), \ldots, (p_m - p^0_m)$, iniciando com termos de segundo grau,
$a_{ij} = \frac{\partial x_i}{\partial p_j}$ e $b_{ij} = \frac{\partial D_i}{\partial p_j}$.

Se definirmos $\displaystyle \frac{dp_i}{dt} = q_i$ e utilizarmos as equações acima,
chegamos em

\begin{equation} \label{dqi}
\begin{aligned}
	& \frac{dq_i}{dt} = \beta_i x^0_i + \somamj \beta_i a_{ij} (p_j - p^0_j) +
	\somamj (\alpha_i a_{ij} + \beta_i b_{ij})q_j + \\
	&+ [(p_1 - p^0_1), \ldots, (p_m - p^0_m);q_1, \ldots, q_m] \espaco i=1, \ldots, m
\end{aligned}
\end{equation}

As equações \ref{qi} e \ref{dqi} formam o sistema que temos que lidar.

\begin{definicao}
	Para que $p^0 = (p^0_1, \ldots, p^0_m)$ seja um ponto de equilíbrio no tempo
	$t_0$, é preciso que

		\[\frac{dp_i}{dt} = 0, \qquad \frac{dq_i}{dt} = 0, \espaco i = 1, \ldots, m\]

\end{definicao}

Portanto, se $p^0$ é um ponto de equilíbrio em $t_0$, temos que

\begin{equation} \label{equicondi}
	\begin{cases}
		\begin{aligned}
			&\alpha_i x^0_i & + & \beta_i X^0_i = 0, \\
			&               &   & \beta_i x^0_i = 0,
		\end{aligned}
	\end{cases}
\end{equation}

que por sua vez, é equivalente à condição

\begin{equation}
	x_i^0 = X^i_0 = 0 \; \text{se} \; \beta_i \neq 0, \qquad x_i^0 = 0 \; \text{se} \; \beta_i = 0.
\end{equation}

Em palavras, as condições necessárias e suficientes para o equilíbrio implicam que

\begin{itemize}
	\item o excesso de demanda por estoques e o excesso de demanda corrente desaparecem no equilíbrio se o bem não é fluxo-orientado;
	\item o excesso de demanda corrente desaparece no equilíbrio se o bem é fluxo-orientado.
\end{itemize}


Vamos redefinir algumas variáveis para enxugar a notação.

\begin{definicao} Supondo ainda que $p^0 = (p^0_1, \ldots, p^0_m)$ é um vetor de preços de equilíbrio,
	a nova notação é dada por:
 \begin{itemize}
 	\item $P_i = p_i - p_i^0$;
	\item $Q_i = \frac{dp_i}{dt} = \frac{dP_i}{dt}$;
	\item $P = (P_1, \ldots, P_m)$;
	\item $Q = (Q_1, \ldots, Q_m)$;
	\item $\A = (\mathit{A_{ij}})$;
	\item $\mathit{A_{ij}} = \beta_i a_{ij}$;
	\item $\B = (\mathit{B_{ij}})$;
	\item $\mathit{B_{ij}} = \alpha_i a_{ij} + \beta_i  b_{ij}$.
 \end{itemize}
\end{definicao}

Para o restante da seção, supomos que $\beta_i \neq 0 \; \forall i = 1, \ldots, m$.

Assim, podemos escrever em notação vetorial o sistema composto por \ref{qi} e \ref{dqi},

\begin{equation} \label{eqnlin}
	\begin{cases}
		\frac{dP}{dt} = Q \\
		\frac{dQ}{dt} = \textbf{A}P + \textbf{B}Q + [P_1, \ldots, P_m; Q_1, \ldots Q_n].
	\end{cases}
\end{equation}

A estabilidade do sistema \ref{eqnlin} é implicada pela estabilidade de sua versão
linearizada,

\begin{equation} \label{eqlin}
	\begin{cases}
		\frac{dP}{dt} = Q \\
		\frac{dQ}{dt} = \textbf{A}P + \textbf{B}Q.
	\end{cases}
\end{equation}

A matriz de coeficientes é, portanto,

\begin{equation} \label{mateqlin}
	\begin{pmatrix}
		\boldsymbol{0} & \boldsymbol{I} \\
		\A & \B
	\end{pmatrix}.
\end{equation}

O polinômio característico de \ref{mateqlin} é

\begin{equation} \label{policar}
	p(\lambda) = |\I \lambda^2 - \B \lambda - \A|.
\end{equation}

Vamos ver alguns teoremas sobre a estabilidade desse sistema.

\begin{teorema} \label{teor}
	Se as matrizes $\A$ e $\B$ são negativas definidas, toda raiz real de \ref{policar}
	é negativa.
\end{teorema}

\begin{proof}
	Seja $\lambda$ uma raiz real de \ref{policar} e seja $M$ um vetor característico real correspondente.
	Isso implica que

	\begin{equation} \label{poliM}
		(M)\lambda^2 - (\B M)\lambda - (\A M) = 0.
	\end{equation}

	Tomando o produto escalar de \ref{poliM} com $M$, temos

	\begin{equation} \label{poliMum}
		(M^T M)\lambda^2 - (M^T \B M)\lambda - (M^T \A M) = 0.
	\end{equation}

	Pela nossa hipótese, todos coeficientes da equação \ref{poliMum} são positivos.
	Portanto, pela regra dos sinais de Descartes, a raiz real $\lambda$ só pode ser negativa.

\end{proof}

\begin{teorema} \label{teor2}
	Se as matrizes $\A$ e $\B$ são negativas definidas e a matriz $\A$ é simétrica,
	toda raiz de \ref{policar} tem parte real negativa.
\end{teorema}

\begin{proof}

	Seja $\lambda = \sigma + \tau i$ uma raiz de \ref{policar} e seja $M$ um vetor
	característico correspondente. Tomando o produto escalar de \ref{poliM} com
	$\Mbarra$, o conjugado de $M$, temos

	\begin{equation} \label{poliMdois}
		(\Mbarra^T M)\lambda^2 - (\Mbarra^T \B M)\lambda - (\Mbarra^T \A M) = 0
	\end{equation}

	Tirando o conjugado dos dois lados, temos

	\begin{equation} \label{poliMtres}
		(M^T \Mbarra) \overline{\lambda}^2 - (M^T \B \Mbarra) \overline{\lambda} - (M^T \A \Mbarra) = 0
	\end{equation}

	Dividindo \ref{poliMdois} por $\lambda$, \ref{poliMtres} por $\overline{\lambda}$ e somando ambos, temos

	\begin{equation}
		(\Mbarra^T M)\sigma - \frac{1}{2}(\Mbarra^T \B M + M^T \B \Mbarra) - (\Mbarra^T \A M) \frac{\sigma}{\sigma^2 + \tau^2} = 0.
	\end{equation}

	Colocando $M = U + iV$, chegamos em

	\begin{equation}
		(\Mbarra^T M)\sigma - (U^T \B U + V^T \B V) - (U^T \A U + V^T \A V) \frac{\sigma}{\sigma^2 + \tau^2} = 0.
	\end{equation}

	Como as matrizes $\A$ e $\B$ são negativas definidas, temos novamente, pela regra
	dos sinais de Descartes, que todos coeficientes são positivos e as raízes $\sigma$
	são positivas.

\end{proof}

\begin{teorema}
	Se a matriz $\B$ é negativa definida e se as matrizes $\A$ e $\B$ são simétricas,
	então toda raiz complexa de \ref{policar} tem parte real negativa.
\end{teorema}

\begin{proof}
	Novamente, seja a raiz $\lambda = \sigma + \tau i$ e $M = U + iV$. Por hipótese, tratamos de
	raízes complexas, ou seja, $\tau \neq 0$. Como $\A$ e $\B$ são simétricas, \ref{poliMdois}
	só possui coeficientes reais. Portanto, $\overline{\lambda}$ é a outra raiz e temos

	\begin{equation}
		\lambda + \overline{\lambda} = 2\sigma = \frac{\Mbarra^T \B M}{\Mbarra^T M} = \frac{U^T \B U + V^T \B V}{\Mbarra^T M},
	\end{equation}

	sendo o último pedaço negativo. Portanto, $\sigma < 0$ e o sistema é estável.
\end{proof}

Note que o teorema \ref{teor} não nos garante que a mera propriedade de matrizes
negativas definidas implica estabilidade. É preciso também que todas as raízes sejam
reais. O exemplo abaixo ilustra isso.

Seja $n = 2$. As matrizes abaixo

\begin{equation}
	\A =
	\begin{pmatrix}
		-4 & 6 \\
		1 & -4
	\end{pmatrix}, \quad
	\B =
	\begin{pmatrix}
		-4 & 1 \\
		6  & -4
	\end{pmatrix}
\end{equation}

são ambas negativas definidas. Montando o polinômio \ref{policar} para
essas matrizes, temos

\begin{equation}
	\lambda^4 + 8\lambda^3 + 18\lambda^2 - 5\lambda + 10 = 0.
\end{equation}

As raízes são $\lambda_{1,2} \approx 0.21290 + 0.65444i$ e $\lambda_{3, 4} \approx -4.2129 \pm 1.8346i$.
Logo, há um par de raízes com coeficiente real positivo e o sistema não é estável.

Sem lançar mão das condições de simetria, os autores não conseguem formular uma condição geral o suficiente
para garantir a estabilidade.

De qualquer maneira, toda a discussão de Clower e Bushaw se baseia no cômputo explícito das
soluções do sistema, ou como eles mesmo chamam, do "primeiro método de Liapunov".
A nossa discussão toda centra-se no segundo método de Liapunov, isto é, da utilização
de uma função auxiliar para identificar as condições de estabilidade. Ao fazer
esse breve comentário sobre Liapunov, os autores provam o teorema \ref{teor2}
utilizando o segundo método. Vamos vê-lo.

\begin{teorema}
	Supondo as mesmas condições de \ref{teor2} ($\A$ simétrica e negativa definida,
	$\B$ negativa definida) e que existe uma função contínua $V(P, Q)$ que
	\begin{enumerate}
		\item zera quando $P = Q = 0$,
		\item é positiva definida e
		\item tem uma derivada total $\frac{dV}{dt} \leq 0$,
	\end{enumerate}

	então o estado de equilíbrio $P = Q = 0$ é estável
\end{teorema}

Antes da demonstração, note que os autores simplesmente definem uma função de Liapunov, como
na definição \ref{funcliapunov} e enunciam o teorema \ref{teorliapunov}.

\begin{proof}
	Definimos a função

	\begin{equation}
		V(P, Q) = Q^T Q - P^T \A P,
	\end{equation}

	que cumpre 1 e 2. Sua derivada (computada utilizando \ref{eqnlin}) é
	$\displaystyle \frac{dV}{dt} = 2Q^T \B Q$, que cumpre 3 por $\B$ ser negativa definida.
\end{proof}

O resto da discussão dos autores trata de seu modelo de estoque-fluxo.
Isso não nos interessa, pois já vimos em qual situação os autores apresentaram
os métodos de Liapunov. Há mais um corolário provado utilizando os mesmos métodos,
na página 339, nota de rodapé 10.

Como comentei na introdução, Clower e Bushaw apenas se valem marginalmente
da função de Liapunov para seu trabalho. Falta apenas ver como Arrow e Hurwicz
utilizam os métodos no seu artigo.

% ----------------------------------------------------------
% PARTE ARROW E HURWICZ
% ----------------------------------------------------------
\part{Arrow e Hurwicz}
% ----------------------------------------------------------

% ---
% Capítulo de Lakatos
% ---
%\chapter{O núcleo duro do programa neowalrasiano} \label{lakatos}
% ---

%\section{Um outro programa de pesquisa}

% ---
% Capítulo de Arrow e Hurwicz
% ---
\chapter{O artigo de Arrow e Hurwicz} \label{arrowhurwicz}
% ---

% ---
\section{Definindo conceitos}
% ---

O objetivo dos autores em seu artigo é preencher algumas lacunas deixadas
pela comunidade científica até então. Isto é, apesar de os economistas à época
estudarem a relação entre a estabilidade dinâmica de Samuelson e a estabilidade
perfeita de Hicks, pouco se sabia sobre \textit{quais} condições garantiam estabilidade.
Assim, era precisa descobrir se a estabilidade era compatível com preferências
convexas, racionais, etc. Essa era o projeto de pesquisa dos novos economistas matemáticos,
como Arrow, Debreu, Hahn, Gale, entre outros.

O trabalho de \citeonline{arrowhurwickz1958} mostrará algumas economias que eram
compatíveis com estabilidade, preenchendo a lacuna acima.

O trabalho começa definindo a notação que será usada.

O sistema dinâmico a ser resolvido pode ser representado como $m$ equações
diferenciais ordinárias:

\begin{subequations}

	\begin{equation} \label{sistemanormal}
	 	\frac{dp_j}{dt} = f_j(p_1, \ldots, p_m) \espaco (j = 1, \ldots, m)
	\end{equation}

	\begin{equation} \label{sistemavetor}
		\frac{dp}{dt} = f(p) \espaco p = (p_1, \ldots, p_m)
	\end{equation}

\end{subequations}

A equação \ref{sistemavetor} é a forma vetorial de \ref{sistemanormal}.

\begin{definicao}
$\solucao$ é uma (trajetória da) solução de \ref{sistemanormal}
(ou \ref{sistemavetor}) se, e só se

\begin{subequations}

	\begin{equation} \label{trajetoria1}
		\frac{d\solucao}{dt} = f[\solucao]
	\end{equation}

	\begin{equation} \label{trajetoria2}
		\Psi(0, \pbola) = \pbola
	\end{equation}

\end{subequations}

\end{definicao}

Note que a condição \ref{trajetoria2} diz apenas que $\pbola$ é o valor
inicial da solução.

\begin{definicao}
	Um ponto $\pbarra \in \R^m$ é um ponto de equilíbrio de \ref{sistemavetor}
	se, e só se

	\begin{equation} \label{ptoequilibrio}
		f(\pbarra) = 0
	\end{equation}

\end{definicao}

\begin{definicao}
	Um ponto de equilíbrio $\pbarra$ é chamado de localmente estável se existe
	uma vizinhança $N(\pbarra)$ tal que para qualquer $\pbola \in N(\pbarra)$,
	toda solução $\solucao$ convergirá para $\pbarra$, isto é

	\begin{equation} \label{estabpequeno}
		\lim_{t \to \infty} \solucao = \pbarra
	\end{equation}

\end{definicao}

Esta é a definição de Samuelson de estabilidade do primeiro tipo no "pequeno".

\begin{definicao}
	Um ponto de equilíbrio $\pbarra$ é chamado de globalmente estável se
	toda solução $\solucao$ (com qualquer $\pbola$ inicial) converge para
	$\pbarra$.
\end{definicao}

Esta é a definição de Samuelson de estabilidade do primeiro tipo no "grande".

Existe a possibilidade de haver múltiplos equilíbrios. Para acomodar o conceito
de estabilidade nesse caso, os autores se referem à seguinte definição.

\begin{definicao}
	Seja $E$ o conjunto de todos pontos de equilíbrio. O sistema \ref{sistemavetor}
	é dito estável se, e só se, para todo $\pbola$ inicial, $\solucao$ converge para algum
	$\pbarra \in E$.
\end{definicao}

Agora os autores dão significado à notação acima.

\begin{definicao}
	O significado econômico das variáveis acima:
	\begin{itemize}
		\item $p_0 = 1$ é o preço do bem "numerário";
		\item $P_1, \ldots, P_m$ são os preços dos outros bens;
		\item $p_1, \ldots, p_m$ são os preços dos outros bens em termos do numerário,
		isto é, $p_1 = P_1/p_0, \ldots, p_m = P_m/p_0$;
	  \item $p = (p_0, p_1, \ldots, p_m)$ é o vetor de preços normalizado;
		\item $\frac{dp_j}{dt} = f_j(p_1, \ldots, p_m)$, (j = 1, \ldots, m), é a
		função de excesso de demanda agregada do bem $j$.
	\end{itemize}
\end{definicao}

O último item é o sistema \ref{sistemanormal} montado acima e define o \textit{processo de ajustamento instantâneo}. Como vimos no \autoref{metzler},
Metzler estuda também o sistema no caso em que as velocidades de ajustamento variam, a saber

\begin{equation} \label{sistemavel}
	\frac{dp_j}{dt} = k_jf_j(p_1, \ldots, p_m), \quad k_j > 0 \espaco (j = 1, \ldots, m)
\end{equation}

em que $k_j$ é a velocidade de ajustamento no mercado $j$. Já sabemos que escolhendo
as unidades de medida convenientes para $k_j$ conseguimos reduzir \ref{sistemavel}
a \ref{sistemanormal}. Portanto, se há estabilidade em \ref{sistemanormal}, a estabilidade
em \ref{sistemavel} é determinada por consequência.

Isso torna-se mais importante ainda dado o teorema \ref{metzler1} de Metzler. %% REFERENCIAR RESULTADO DE METZLER
Portanto, temos o seguinte teorema

\begin{teorema}
	Se o sistema \ref{sistemanormal} é estável $\implies$ o sistema \ref{sistemavel}
	é estável $\implies$ há estabilidade perfeita no sentido de Hicks no equilíbrio.
\end{teorema}

\begin{proof}
Veja o \autoref{metzler} para a prova.
\end{proof}

Existem $n$ unidades (firmas, indivíduos) e cada uma delas maximiza
alguma entidade (lucro, utilidade) que está em função dos preços.
Em competição perfeita, os preços são exógenos.

O artigo trata apenas dos casos de trocas puras (as menções a economias de produção
restringem-se aos rodapés). Cada indivíduo $i$ se depara com o seguinte problema

\begin{equation} \label{otimizacao}
\begin{aligned}
& \text{max}
& & u^i(X^i_0, X^i_1, \ldots, X^i_m) \\
& \text{sujeito a}
& & \sum^m_{k = 0} p_k\Xik = \sum^m_{k = 0} p_k\Xbolaik, \\
&&& \Xik \geq 0
&&&&& (k = 0, 1, \ldots, m)
\end{aligned}
\end{equation}

em que $\Xik$ é a quantidade consumida e
$\Xbolaik$ é a dotação inicial possuída pelo indivíduo $i$ do bem $k$.

A solução do problema \ref{otimizacao} pode ser escrita como

\begin{equation}
	\Xchapeuik =  \Xbolaik + f^i_k(p_1, \ldots, p_m) \espaco (k = 0, 1, \ldots, m)
\end{equation}

em que $f^i_k$ é a função excesso de demanda do indivíduo $i$ pelo bem $k$.
A função excesso de demanda \textbf{agregada} do bem $k$ é definida por

\begin{equation}
	f_k(p_1, \ldots, p_m) = \somani  f^i_k(p_1, \ldots, p_m) \espaco
	(k = 0, 1, \ldots, m).
\end{equation}

A existência do equilíbrio competitivo significa que existe um vetor

\begin{equation}
	\pbarra = (\pbarra_1, \ldots, \pbarra_m)
\end{equation}

tal que as duas condições abaixo são cumpridas

\begin{subequations}
\begin{equation}
	f_j(\pbarra) = 0 \espaco (j = 1, \ldots, m)
\end{equation}

\begin{equation} \label{equilibrio}
	\pbarra \geqslant 0.
\end{equation}
\end{subequations}

Note que a desigualdade em \ref{equilibrio} é diferente da desigualdade em
\ref{otimizacao}. Falarei um pouco sobre essa diferença de notações.

Pegue dois vetores $u$ e $v$, com $m$ componentes cada um. A desigualdade
$u \geq v$ significa que $u_i \geq v_i$ para todo $i = 1, \ldots, m$,
mas $u_j > v_j$ para algum $j$. Já a desigualdade $u \geqslant v$ significa
apenas que $u_i \geq v_i$ para todo $i = 1, \ldots, m$. Portanto, $u \geq v$ é uma condição mais forte que $u \geqslant v$.

Os autores concluem a introdução e começam a mostrar os resultados.
Meu interesse é apresentar apenas a parte I do artigo, em que o segundo método
de Liapunov é usado.

\section{O caso da ausência de trocas no equilíbrio}

Antes de continuar a apresentação do artigo, é importante fazer um comentário
sobre a nota de rodapé 21 na página 530. O rodapé é esse: "See Arrow and
Hurwicz [5]. This is a special case of Liapounoff's 'second method' for
proving stability". A referência de número 5 é o trabalho de
\citeonline{arrowhurwicz1951}. Na página 5 desse artigo, os autores definem uma função D,
que é "proportional to the distance in the $(X, Y)$ space to the saddle-point
$(\overline{X}, \overline{Y})$". Em outras palavras, é uma aplicação de
Liapunov.

Ademais, na bibliografia do artigo há o livro de \citeonline{lefschetz1948}. O
livro discute \textit{extensamente} os métodos de Liapunov. Portanto, é
suficiente dizer que a comunidade de economia já tinha conhecimento dos métodos, antes da publicação de
Clower e Bushaw.

Continuemos. Seja $D$ a distância entre um ponto $p$ para algum equilíbrio
$\pbarra$.

\begin{equation}
	D^2 = \sum^m_{k=0}(p_k - \pbarra_k)^2 = \somamj(p_j - \pbarra_j)^2.
\end{equation}

A última igualdade vem do fato que $p_0 = \pbarra_0 = 1$. Mostraremos que, sob
certas condições, $\frac{dD}{dt} < 0$ e, com isso, $p \to \pbarra$ quando $t \to \infty$. Definimos a distância $V$, tal que

\begin{equation}
	V = \frac{1}{2}D^2
\end{equation}

e derivamos em relação ao tempo, obtendo

\begin{equation} \label{derivadaV1}
	\frac{dV}{dt} =  \somamj(p_j - \pbarra_j)(\frac{dp_j}{dt}) = \somamj(p_j - \pbarra_j)x_j.
\end{equation}

Aqui, $x_k = \frac{dp_k}{dt} = f_k(p_1, \ldots, p_m)$, com $k = 0, 1, \ldots,
m$. Novamente, $x_k$ representa o excesso de demanda do bem $k$. Para definir
o excesso de demanda do indivíduo $i$ pelo bem $k$ usamos $\xik$, de tal forma
que

\begin{equation}
	\xik = \Xchapeuik - \Xbolaik, \espaco x_k = \somani\xik.
\end{equation}

Fazendo as devidas substituições em \ref{derivadaV1}, obtemos

\begin{equation} \label{derivadaV2}
	\begin{aligned}
	\frac{dV}{dt} = \somamj p_jx_j - \somamj \pbarra_jx_j = \somamj p_j \somani \xij - \somamj \pbarra_jx_j = \\
	= \somani(\somamj p_j\xij) - \somamj \pbarra_jx_j
\end{aligned}
\end{equation}

Pegue a restrição orçamentária em \ref{otimizacao},

\begin{equation} \label{restricao}
	\sum^m_{k=0} p_k\xik = 0
\end{equation}

e a substitua em \ref{derivadaV2} (utilizando o fato de que $\pbarra_0 = p_0 = 1$). Assim, obtemos

\begin{equation}
	\frac{dV}{dt} = \somani(-p_0x^i_0) - \somamj\pbarra_j(\somani \xij) = - \somani(\sum^m_{k=0}\pbarra_k \xik).
\end{equation}

\begin{lema}
	Suponha que cada indivíduo satisfaça o axioma fraco das preferências reveladas \footnote{Confira \citeonline{mwg1995}.} Então,
	\begin{equation}
		\xbarraik = 0 \quad \forall i = 1, \ldots, n \  \text{e} \ \forall k = 0, \ldots, m \implies - \somamj(p_j - \pbarra_j)\xij = \sum^m_{k=0}\pbarra_k\xik > 0
	\end{equation}
\end{lema}

\begin{proof}
	Por definição, $\xbarra^i \equiv (\xbarra^i_0, \ldots, \xbarra^i_m)$ maximiza
	a utilidade de $i$ sujeito a $\Pbarra \cdot \xbarra^i$. No entanto, como
	$\xbarra^i = 0$ (por hipótese), então $P \cdot \xbarra^i = 0$. Portanto, o
	antecedente do axioma da preferência revelada se mantém na forma
	$P \cdot \xbarra^i \leq P \cdot x^i$ pela restrição orçamentária $P \cdot x^i$.
	Se segue então que $P \cdot x^i > P \cdot \xbarra^i$, a não ser que $\xbarra^i = x^i$.
	Isso fornece $\Pbarra \cdot x^i > 0$ a não ser que $x^i = 0$, pois $\Pbarra \cdot \xbarra^i = 0$.
	Dividindo ambos os lados de $\Pbarra \cdot x^i > 0$ por $\Pbarra_0$, nós obtemos o lema.
\end{proof}

Portanto, se segue que $\frac{dV}{dt} < 0$ a não ser que $x^i = 0$, isto é,
$x^i = \xbarra^i$ para todo $i$.

Portanto, nos falta ver o caso em que $x^i = \xbarra^i$ para todo $i$. Caso isso ocorra,
temos que

\begin{equation}
	u^i_k(x^i) = u^i_k(\xbarra^i) \espaco \text{para todo} \; i, k,
\end{equation}

em que $u^i_k \equiv \frac{\partial u}{\partial \xik}$.

Por sua vez, no máximo individual (isto é, com toda restrição orçamentária gasta)

\begin{equation}
	p_j = \frac{u^i_j}{u^i_0}.
\end{equation}

Portanto, temos que $\frac{dV}{dt} < 0$ a não ser que

\begin{equation}
	p_j = \frac{u^i_j}{u^i_0} \pbarra_j,
\end{equation}

isto é, quando

\begin{equation}
	p_j = \pbarra_j \espaco j = 1, \ldots, m
\end{equation}

de tal forma que o sistema está em equilíbrio. Assim

\begin{equation}
	\lim_{t \to \infty} \solucao = \pbarra \espaco \forall \; \pbola.
\end{equation}

Como estabilidade global implica a unicidade do equilíbrio, nós temos o

\begin{teorema}
	Suponha que toda função excesso de demanda indivídual é contínua e nenhum indivíduo
	está saturado (isto é, a restrição em \ref{otimizacao} é com igualdade e não desigualdade).
	Deixe $\xbarra^i = 0$ para todo $i = 1, \ldots, m$. Então, o processo de ajustamento
	instantâneo é estável em trocas puras. Além disso, possui um único vetor de preços de equilíbrio.
\end{teorema}

Após isso, Arrow e Hurwicz procuram vários casos em que a condição $\xbarra^i = 0$
é satisfeita. Isso não interessa ao trabalho, pois a intenção era apenas mostrar
em que partes eles utilizaram o teorema de Liapunov, em contraste com os usos
de Clower e Bushaw.

Por fim, no apêndice da parte I os autores continuam usando extensamente os métodos de
Liapunov. Aqui, ao invés de tratar de processos de ajustamento instantâneo, os autores
trabalham com processos de ajustamento com defasagem entre mercados. Novamente,
não interessa ao trabalho, pois o resultado principal já foi mostrado.

% ----------------------------------------------------------
% Finaliza a parte no bookmark do PDF
% para que se inicie o bookmark na raiz
% e adiciona espaço de parte no Sumário
% ----------------------------------------------------------
\phantompart

% ---
% Conclusão
% ---
\chapter{Conclusão}
% ---

Para Weintraub e outros historiadores do pensamento econômico, a história é
apenas uma narrativa criada pelo historiador e há inúmeras versões dela.
Segundo eles, narrativas que se pretendem mostrar como uma sequência de
acontecimentos que levam inevitavelmente à Verdade e ao Progresso trariam
consigo vícios e distorções dos acontecimentos históricos. Essa é a posição
da Sociologia do Conhecimento Científico. Weintraub tenta mostrar
que a \textit{survey} de \citeonline{negishi1962} é uma dessas narrativas.
O autor seleciona o artigo de \citeonline{bushawclower1954} e o retrata como se
fosse um artigo fundamental para a discussão sobre estabilidade em equilíbrio
walrasiano da época. Ele segue a crítica argumentando que o artigo, apesar de
introduzir os métodos de Liapunov na teoria econômica, foi ignorado por
\citeauthoronline{negishi1962}.

O que mostrei em meu trabalho foi a irresponsabilidade de Weintraub em retratar
dessa maneira o artigo, pois a) os métodos de Liapunov já eram conhecidos na comunidade
acadêmica de economia antes do artigo de Clower e Bushaw, b) Clower e Bushaw não faziam
parte do mesmo programa de pesquisa dos economistas matemáticos (Arrow, Debreu, Gale, etc.) e
não contribuíram para a expansão do conhecimento sobre estabilidade em equilíbrio walrasiano e c)
Clower e Bushaw utilizam apenas marginalmente os métodos de Liapunov.

Diante disso, a tese de Weintraub fica enfraquecida e pouco esclarece a discussão
da época. Apesar disso, o resto do livro de Weintraub traz alguns \textit{insights} importantes.
Sua tese sobre os métodos de Liapunov não foi um deles.

% ----------------------------------------------------------
% ELEMENTOS PÓS-TEXTUAIS
% ----------------------------------------------------------
\postextual
% ----------------------------------------------------------

% ----------------------------------------------------------
% Referências bibliográficas
% ----------------------------------------------------------
\bibliography{bibliografia}

% ----------------------------------------------------------
% Glossário
% ----------------------------------------------------------
%
% Consulte o manual da classe abntex2 para orientações sobre o glossário.
%
%\glossary

% ----------------------------------------------------------
% Apêndices
% ----------------------------------------------------------

% ---
% Inicia os apêndices
% ---
% \begin{apendicesenv}

% Imprime uma página indicando o início dos apêndices
% \partapendices

% ----------------------------------------------------------
% \chapter{Quisque libero justo}
% ----------------------------------------------------------

%\end{apendicesenv}
% ---


% ----------------------------------------------------------
% Anexos
% ----------------------------------------------------------

% ---
% Inicia os anexos
% ---
%\begin{anexosenv}

% Imprime uma página indicando o início dos anexos
%\partanexos

%\end{anexosenv}

\end{document}
